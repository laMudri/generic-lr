\documentclass[fleqn,aspectratio=169]{beamer}
\usetheme{Rochester}
\setbeamertemplate{navigation symbols}{}

\usepackage{booktabs}
\usepackage{subcaption}

\usepackage{stmaryrd}
\usepackage{mathpartir}
\usepackage{amssymb}
\usepackage{mathrsfs}
\usepackage{cmll}
\usepackage{xcolor}
\usepackage{makecell}
\usepackage{tikz-cd}
\usepackage{nccmath}
\usepackage{mathtools}
\usepackage{graphicx}
\usepackage[style=alphabetic,citestyle=authoryear]{biblatex}
\addbibresource{../generic-lr.bib}

\definecolor{use}{HTML}{008000}
\newcommand\gr[1]{{\color{use}#1}}
\newcommand\grctx[1]{\gr{\mathcal{#1}}}
\newcommand\grP{\grctx P}
\newcommand\grQ{\grctx Q}
\newcommand\grR{\grctx R}
\newcommand\grctxsub[2]{\grctx{#1}_{\gr{#2}}}
\newcommand\grPe{\grctxsub P e}
\newcommand\grPf{\grctxsub P f}
\newcommand\grQe{\grctxsub Q e}
\newcommand\grQf{\grctxsub Q f}
\newcommand\sem[1]{\left\llbracket{#1}\right\rrbracket}
\newcommand\ps{\mathit{ps}}
\newcommand\qs{\mathit{qs}}
\newcommand\rs{\mathit{rs}}
\newcommand\dotto{\mathrel{\dot\to}}
\newcommand\dottimes{\mathbin{\dot\times}}
\newcommand\wand{\mathrel{\mathord{-}\hspace{-0.75ex}*}}
\newcommand\sep{\mathbin{*}}
\newcommand\env[1]{(#1\mathrm{-Env})}
\newcommand\thinningN{\mathrm{Thinning}}
\newcommand\thinning[2]{\thinningN~#1~#2}
\newcommand\V{\mathcal V}
\newcommand\C{\mathcal C}
\newcommand\sqin{\mathrel{\mathrlap{\sqsubset}{\mathord{-}}}}
\newcommand\sqni{\mathrel{\mathrlap{\sqsupset}{\mathord{-}}}}
\newcommand\subres{=}


\newcommand{\ann}[2]{#1 : #2}
\newcommand{\emb}[1]{\underline{#1}}
\newcommand{\base}[0]{\iota}
\newcommand{\fun}[2]{#1 \multimap #2}
\newcommand{\lam}[2]{\lambda #1.~#2}
\newcommand{\app}[2]{#1~#2}
\newcommand{\excl}[2]{\oc_{\gr{#1}} #2}
\newcommand{\bang}[1]{\left[#1\right]}
\newcommand{\bm}[4]{\if\newelims0%
\operatorname{bm}_{#1}(#2, \bind{#3}{#4})%
\else%
\mathrm{let}~\bang{#3} = #2~\mathrm{in}~#4%
\fi}
\newcommand{\tensorOne}[0]{1}
\newcommand{\unit}[0]{(\mathbin{_\otimes})}
\newcommand{\del}[3]{\if\newelims0%
\operatorname{del}_{#1}(#2, #3)%
\else%
\mathrm{let}~\unit = #2~\mathrm{in}~#3%
\fi}
\newcommand{\tensor}[2]{#1 \otimes #2}
\newcommand{\ten}[2]{(#1 \mathbin{_{\otimes}} #2)}
\newcommand{\prm}[5]{\if\newelims0%
\operatorname{pm}_{#1}(#2, \bind{#3, #4}{#5})%
\else%
\mathrm{let}~\ten{#3}{#4} = #2~\mathrm{in}~#5%
\fi}
\newcommand{\withTOne}[0]{\top}
\newcommand{\eat}[0]{(\mathbin{_{\with}})}
\newcommand{\withT}[2]{#1 \with #2}
\newcommand{\wth}[2]{(#1 \mathbin{_\with} #2)}
\newcommand{\proj}[2]{\operatorname{proj}_{#1} #2}
\newcommand{\sumTZero}[0]{0}
\newcommand{\exf}[2]{\operatorname{ex-falso} #2}
\newcommand{\sumT}[2]{#1 \oplus #2}
\newcommand{\inj}[2]{\operatorname{inj}_{#1} #2}
\newcommand{\cse}[6]{\if\newelims0%
\operatorname{case}_{#1}(#2, \bind{#3}{#4}, \bind{#5}{#6})%
\else%
\mathrm{case}~#2~\mathrm{of}~\inj{L}{#3} \mapsto #4
                         ~;~ \inj{R}{#5} \mapsto #6%
\fi}
\newcommand{\listT}[1]{\operatorname{List} #1}
\newcommand{\nil}[0]{[]}
\newcommand{\cons}[2]{#1 :: #2}
\newcommand{\fold}[5]{\operatorname{fold} #1~#2~(#3,#4.~#5)}

\usetikzlibrary{positioning}
% \usetikzlibrary{decorations.markings}
\usetikzlibrary{arrows}

\newsavebox{\pullback}
\sbox\pullback{%
  \begin{tikzpicture}%
    \draw (0,0) -- (1em,0em);%
    \draw (1em,0em) -- (1em,1em);%
  \end{tikzpicture}%
}

\def\newelims{1}
\def\multnotation{1}
%\ifx\newelims\undefined
  \def\newelims{0}
\fi
\ifx\multnotation\undefined
  \def\multnotation{0}
\fi

\newcommand{\fixme}[1]{{\color{red}\underline{\em{#1}}}}
\newcommand{\fixmeBA}[1]{{\color{red}{[Bob:{\em #1}]}}}


\newcommand{\bind}[2]{{#2}\{{#1}\}}
\newcommand{\ctx}[2]{%
  \if\multnotation0%
  #1^{\resctx{#2}}%
  \else%
  \resctx{#2}#1%
  \fi
}
\newcommand{\ctxvar}[3]{%
  \if\multnotation0%
  #1 \stackrel{\rescomment{#3}}: #2%
  \else%
  \rescomment{#3}#1 : #2%
  \fi%
}
\definecolor{res}{HTML}{008000}
\definecolor{resperm}{HTML}{008000}
\newcommand{\rescomment}[1]{{\color{res}#1}}
\newcommand{\rescommentperm}[1]{{\color{resperm}#1}}
\newcommand{\resctx}[1]{\rescomment{\mathcal{#1}}}
\newcommand{\resctxperm}[1]{\rescommentperm{\mathcal{#1}}}
\newcommand{\resmat}[1]{\rescomment{#1}}


\newcommand{\ann}[2]{#1 : #2}
\newcommand{\emb}[1]{\underline{#1}}

\newcommand{\base}[0]{\iota}

\newcommand{\fun}[2]{#1 \multimap #2}
\newcommand{\lam}[2]{\lambda #1.~#2}
\newcommand{\app}[2]{#1~#2}

\newcommand{\excl}[2]{\oc_{\rescomment{#1}} #2}
\newcommand{\bang}[1]{\left[#1\right]}
\newcommand{\bm}[4]{\if\newelims0%
\operatorname{bm}_{#1}(#2, \bind{#3}{#4})%
\else%
\mathrm{let}~\bang{#3} = #2~\mathrm{in}~#4%
\fi}

\newcommand{\tensorOne}[0]{1}
\newcommand{\unit}[0]{(\mathbin{_\otimes})}
\newcommand{\del}[3]{\if\newelims0%
\operatorname{del}_{#1}(#2, #3)%
\else%
\mathrm{let}~\unit = #2~\mathrm{in}~#3%
\fi}

\newcommand{\tensor}[2]{#1 \otimes #2}
\newcommand{\ten}[2]{(#1 \mathbin{_{\otimes}} #2)}
\newcommand{\prm}[5]{\if\newelims0%
\operatorname{pm}_{#1}(#2, \bind{#3, #4}{#5})%
\else%
\mathrm{let}~\ten{#3}{#4} = #2~\mathrm{in}~#5%
\fi}

\newcommand{\withTOne}[0]{\top}
\newcommand{\eat}[0]{(\mathbin{_{\with}})}

\newcommand{\withT}[2]{#1 \with #2}
\newcommand{\wth}[2]{(#1 \mathbin{_\with} #2)}
\newcommand{\proj}[2]{\operatorname{proj}_{#1} #2}

\newcommand{\sumTZero}[0]{0}
\newcommand{\exf}[2]{\operatorname{ex-falso} #2}

\newcommand{\sumT}[2]{#1 \oplus #2}
\newcommand{\inj}[2]{\operatorname{inj}_{#1} #2}
\newcommand{\cse}[6]{\if\newelims0%
\operatorname{case}_{#1}(#2, \bind{#3}{#4}, \bind{#5}{#6})%
\else%
\mathrm{case}~#2~\mathrm{of}~\inj{L}{#3} \mapsto #4
                         ~;~ \inj{R}{#5} \mapsto #6%
\fi}

\newcommand{\listT}[1]{\operatorname{List} #1}
\newcommand{\nil}[0]{[]}
\newcommand{\cons}[2]{#1 :: #2}
\newcommand{\fold}[5]{\operatorname{fold} #1~#2~(#3,#4.~#5)}


\newcommand{\typed}[1]{\mathit{#1t}}
\newcommand{\resourced}[1]{\mathit{#1r}}


\newcommand{\lvar}{\mathrel{\mathrlap{\sqsupset}{\mathord{-}}}}
\newcommand{\sem}[1]{\left\llbracket #1 \right\rrbracket}

%\def\tobar{\mathrel{\mkern3mu  \vcenter{\hbox{$\scriptscriptstyle+$}}%
%                    \mkern-12mu{\to}}}
\newcommand\tobar{\mathrel{\ooalign{\hfil$\mapstochar\mkern5mu$\hfil\cr$\to$\cr}}}


\newcommand{\subres}{\trianglelefteq}
\newcommand{\subrctx}[2]{{#1} \subres {#2}}
\makeatletter
\newcommand{\subst}[2][]{\ext@arrow 0359\Rightarrowfill@{#1}{#2}}
\makeatother


\newenvironment{eqns}{\begin{array}{r@{\hspace{0.3em}}c@{\hspace{0.3em}}l}}{\end{array}}


\newcommand{\mat}[1]{\mathbf{#1}}
\newcommand{\vct}[1]{\mathbf{#1}}
\DeclarePairedDelimiter\basis{\langle}{\rvert}


\newcommand{\name}{\ensuremath{\lambda \resctxperm R}}


\DeclareMathOperator\kit{Kit}
\newcommand{\kitrel}{\mathbin{\blacklozenge}}


\DeclareMathOperator\id{id}
\DeclareMathOperator\inl{inl}
\DeclareMathOperator\inr{inr}


\newcommand{\zero}{\ensuremath{\rescomment 0}}
\newcommand{\linear}{\ensuremath{\rescomment 1}}
\newcommand{\unrestricted}{\ensuremath{\rescomment \omega}}

\newcommand{\unused}{\ensuremath{\rescomment{\mathit{unused}}}}
\newcommand{\true}{\ensuremath{\rescomment{\mathit{true}}}}
\newcommand{\valid}{\ensuremath{\rescomment{\mathit{valid}}}}

\newcommand\dotto{\mathrel{\dot\to}}
\newcommand\dottimes{\mathbin{\dot\times}}
\newcommand\wand{\mathrel{\mathord{-}\hspace{-0.75ex}*}}
\newcommand\sep{\mathbin{*}}

\newcommand{\aval}[1]{\gr{\mathrm{#1}}}

% Beamer hacks

\makeatletter

% Detect mode. mathpalette is used to detect the used math style
\newcommand<>\Alt[2]{%
  \begingroup
  \ifmmode
  \expandafter\mathpalette
  \expandafter\math@Alt
  \else
  \expandafter\make@Alt
  \fi
  {{#1}{#2}{#3}}%
  \endgroup
}

% Un-brace the second argument (required because \mathpalette reads the three arguments as one
\newcommand\math@Alt[2]{\math@@Alt{#1}#2}

% Set the two arguments in boxes. The math style is given by #1. \m@th sets \mathsurround to 0.
\newcommand\math@@Alt[3]{%
  \setbox\z@ \hbox{$\m@th #1{#2}$}%
  \setbox\@ne\hbox{$\m@th #1{#3}$}%
  \@Alt
}

% Un-brace the argument
\newcommand\make@Alt[1]{\make@@Alt#1}

% Set the two arguments into normal boxes
\newcommand\make@@Alt[2]{%
  \sbox\z@ {#1}%
  \sbox\@ne{#2}%
  \@Alt
}

% Place one of the two boxes using \rlap and place a \phantom box with the maximum of the two boxes
\newcommand\@Alt[1]{%
  \alt#1%
  {\rlap{\usebox0}}%
  {\rlap{\usebox1}}%
  \setbox\tw@\null
  \ht\tw@\ifnum\ht\z@>\ht\@ne\ht\z@\else\ht\@ne\fi
  \dp\tw@\ifnum\dp\z@>\dp\@ne\dp\z@\else\dp\@ne\fi
  \wd\tw@\ifnum\wd\z@>\wd\@ne\wd\z@\else\wd\@ne\fi
  \box\tw@
}

\newcommand{\ctxvar}[3]{\gr{#3}#1 : #2}
\newcommand\ctxvarh[3]{\ctxvar{#1}{#2}{\uncover<2>{#3}}}

\makeatother

\title{From Substitution to Semantics for a Family of Substructural Type Systems}
\author{James Wood\inst{1} \and Bob Atkey\inst{1}}
\institute{\inst{1}University of Strathclyde}
\date{VEST, June 2020}

\begin{document}
\setlength{\abovedisplayskip}{0pt}
\frame{\titlepage}
\begin{frame}
  \frametitle{Context}
  \begin{center}
    \begin{tikzcd}[ampersand replacement=\&, sep=large]
      \& |[label=center:\clap{McB05$^{1}$}]|{\phantom M}
      \arrow[ld, "\textrm{linearity}"'] \arrow[rd, "\textrm{semantics}"] \& \\
      |[label=center:\clap{WA20$^{2}$}]|{\phantom M} \arrow[rd] \&
      \& |[label=center:\clap{AACMM18/20$^{3}$}]|{\phantom M} \arrow[ld] \\
      \& |[label=center:\clap{WIP}]|{\phantom M} \arrow[uu, phantom, "\usebox\pullback" rotate=135, very near start] \&
    \end{tikzcd}
  \end{center}
  \begin{itemize}
    \item Linearity independent of binding (de Bruijn indices)
    \item Only one traversal over the syntax
      % Hacks:
      \phantom{\footfullcite{rensub05}
      \footfullcite{linear-metatheory}\footfullcite{AACMM20}}
  \end{itemize}
\end{frame}
\begin{frame}
  \frametitle{Idea --- stability under structurality \footfullcite{AACMM20}}
  Two parts:
  \begin{enumerate}
    \item Consolidate all traversals over syntax (e.g, simultaneous renaming,
      simultaneous substitution, NbE, printing) into a single generic traversal.
    \item Build typing rules from small building blocks, so that they admit a
      generic semantics by construction.
  \end{enumerate}
\end{frame}
\begin{frame}
  \frametitle{Starter --- substitution via kits}
  \begin{columns}
    \begin{column}{0.5\textwidth}
      \begin{block}{Renaming}
        \[
          \inferrule*
          {\forall A.~A \in \Gamma \to {\color{red}A \in \Delta}}
          {\forall A.~A \dashv \Gamma \to A \dashv \Delta}
        \]
      \end{block}
    \end{column}
    \begin{column}{0.5\textwidth}
      \begin{block}{Substitution}
        \[
          \inferrule*
          {\forall A.~A \in \Gamma \to {\color{red}A \dashv \Delta}}
          {\forall A.~A \dashv \Gamma \to A \dashv \Delta}
        \]
      \end{block}
    \end{column}
  \end{columns}

  \let\oldmathindent=\mathindent
  \setlength{\mathindent}{0pt}
  \begin{flalign*}
    &\mathrm{subst}~\sigma~(\mathrm{lam}~M) =
    \mathrm{lam}~(\mathrm{subst}~\only<1>{\wn}\only<2->{(\mathrm{bind}~\sigma)}~M)
    &&
    \begin{cases}
      \sigma : \forall A.~A \in \Gamma \to A \dashv \Delta \\
      M : Y \dashv \Gamma, X \\
      \only<1>{\wn}\only<2->{\mathrm{bind}~\sigma}
      : \forall A.~A \in X, \Gamma \to A \dashv X, \Delta
    \end{cases}
    \uncover<2->{
    \\
    &\mathrm{bind}~\sigma~\mathrm{new} =
      \only<2>{\wn}
      \only<3->{\mathrm{var}~\mathrm{new}} &&&
    \\
    &\mathrm{bind}~\sigma~(\mathrm{old}~i) =
      \only<2-3>{\wn}
      \only<4->{\mathrm{rename}~\rho~(\sigma~i)}
      &&
      \only<4->{
        \begin{cases}
          \rho : \forall A.~A \in \Delta \to A \in X, \Delta
        \end{cases}
      }
    }
  \end{flalign*}
  \setlength{\mathindent}{\oldmathindent}
\end{frame}
\begin{frame}
  \frametitle{Generic syntactic traversal}
  Generalise over renaming and substitution.
  \[
    \inferrule*[right=$\mathrm{trav}$]
    {\mathrm{Kit}~\V \\ \forall A.~A \in \Gamma \to {\color{red}\V~A~\Delta}}
    {\forall A.~A \dashv \Gamma \to A \dashv \Delta}
  \]
  The $\mathrm{Kit}$ contains $\mathit{kit.tm}$, $\mathit{kit.vr}$, and
  $\mathit{kit.str}$.
  \begin{flalign*}
    &\mathrm{trav}~\sigma~(\mathrm{var}~i) = \mathit{kit.tm}~(\sigma~i)
    &&
    \begin{cases}
      \mathit{kit.tm} : \forall A, \Gamma.~\V~A~\Gamma \to A \dashv \Gamma
    \end{cases}
    \\
    &\mathrm{trav}~\sigma~(\mathrm{lam}~M) =
    \mathrm{lam}~(\mathrm{trav}~(\mathrm{bind}~\sigma)~M)
    &&
    \begin{cases}
      \mathrm{bind}~\sigma : \forall A.~A \in X, \Gamma \to \V~A~(X, \Delta)
    \end{cases}
    \\
    &\mathrm{bind}~\sigma~\mathrm{new} = \mathit{kit.vr}~\mathrm{new}
    &&
    \begin{cases}
      \mathit{kit.vr} : \forall A, \Gamma.~A \in \Gamma \to \V~A~\Gamma
    \end{cases}
    \\
    &\mathrm{bind}~\sigma~(\mathrm{old}~i) =
      \mathit{kit.str}~\rho~(\sigma~i)
      &&
      \begin{cases}
        \mathit{kit.str} : \Gamma \subseteq \Delta \to
        \V~A~\Gamma \to \V~A~\Delta \\
        \rho : \forall A.~A \in \Delta \to A \in X, \Delta
      \end{cases}
  \end{flalign*}
\end{frame}
\begin{frame}
  \frametitle{Generic semantic traversal}
  \begin{itemize}
    \item We have the following fundamental lemma of semantics:
      \begin{mathpar}
        \inferrule*[right=$\mathrm{sem}$]
        {\mathrm{Semantics}~\V~\C
          \\ \only<1>{\overbrace{\forall A.~A \in \Gamma \to \V~A~\Delta}
            ^{\textrm{environment}}}
          \only<2->{\Gamma \stackrel{\V}{\Rightarrow} \Delta}}
        {{\underbrace{\forall A.~A \dashv \Gamma \to \C~A~\Delta}
            _{\textrm{traversal}}}}
        \and
        \uncover<2->{\Gamma \stackrel{\V}{\Rightarrow} \Delta =
          \forall A.~A \in \Gamma \to \V~A~\Delta}
      \end{mathpar}
    \item $\mathrm{Semantics}~\V~\C$ contains:
          \qquad \uncover<3->{$(\square \mathcal T)~\Gamma =
          \forall \Delta.~\Gamma \subseteq \Delta \to \mathcal T~\Delta$}
      \begin{itemize}
        \item A proof that $\V$ is stable under structurality
        \item Ways to interpret term constructors semantically, generic in
          context:
          \begin{mathpar}
            \sem{\mathrm{var}}
            : \forall[~\V~A \dotto \C~A~]
            \and
            \sem{\mathrm{app}}
            : \forall[~\square(\C~(A \to B)) \dottimes \square(\C~A) \dotto
            \C~B~]
            \and
            \sem{\mathrm{lam}}
            : \forall[~\square(\V~A \dotto \C~B) \dotto \C~(A \to B)~]
            \and
            \dots
          \end{mathpar}
      \end{itemize}
  \end{itemize}
\end{frame}
\begin{frame}
  \frametitle{Generic notion of syntax}
  \begin{itemize}
    \item A type system can:
      \begin{enumerate}
        \item Offer a multitude of term formers. e.g, \TirName{App},
          \TirName{Lam}, \ldots
        \item For each term former, require 0 or more premises. $\dottimes$,
          $\dot 1$
        \item For each premise, maybe bind variables. $\square$, $\V$
      \end{enumerate}
    \item Variables are a special case.
    \item Example descriptions:
      \begin{itemize}
        \item \phantom{.}$\TirName{App}_{A,B}$: $(A \to B) \times A \Longrightarrow B$
        \item \phantom{.}$\TirName{Lam}_{A,B}$: $(A \vdash B) \Longrightarrow (A \to B)$
      \end{itemize}
  \end{itemize}
\end{frame}
\begin{frame}
  \frametitle{Generic generic semantic traversal}
  If we are given a $\V$-value for each newly bound variable in $\Gamma$, we can produce a computation.
  %For each of the newly bound variables $\Gamma$, we expect a $\V$-value in the
  %result context.
  \begin{flalign*}
    &\mathrm{Kripke}~\V~\C~\Gamma~A =
    \square((\Gamma \stackrel{\V}{\Rightarrow} {-}) \dotto \C~A)
  \end{flalign*}

  Let $d$ be the description of a type system.
  $\sem d$ is one layer of its syntax.

  \begin{flalign*}
    &\sem{\mathrm{var}} : \forall[~\V~A \dotto \C~A~]
    \\
    &\sem{\mathrm{con}} :
    \forall[~\sem d\,(\mathrm{Kripke}~\V~\C)~A \dotto \C~A~]
  \end{flalign*}
  Given $\sem{\mathrm{var}}$ and $\sem{\mathrm{con}}$, we can produce a similar
  traversal to before.
  \begin{flalign*}
    &\mathrm{sem}~\sigma~(\mathrm{var}~i) = \sem{\mathrm{var}}\,(\sigma~i)
    \\
    &\mathrm{sem}~\sigma~(\mathrm{con}~t) =
    \sem{\mathrm{con}}\,(\mathrm{map}~(\mathrm{bind}~\sigma)~t)
  \end{flalign*}
\end{frame}
\begin{frame}
  \frametitle{Example derivation}
  \begin{flalign*}
    & P = \tensor{(\fun{A}{B})}{A} \\\\
    & \inferrule*{
      \inferrule*{
        \inferrule*{ }
        {\ctxvarh{p}{P}{1} \vdash p : P}
        \\
        \inferrule*{
          \inferrule*{ }
          {\ctxvarh{p}{P}{0}, \ctxvarh{f}{\fun{A}{B}}{1}, \\\\ \ctxvarh{x}{A}{0}
            \vdash f : \fun{A}{B}}
          \\
          \inferrule*{ }
          {\ctxvarh{p}{P}{0}, \ctxvarh{f}{\fun{A}{B}}{0}, \\\\ \ctxvarh{x}{A}{1}
            \vdash x : A}
        }
        {\ctxvarh{p}{P}{0}, \ctxvarh{f}{\fun{A}{B}}{1}, \ctxvarh{x}{A}{1}
          \vdash \app{f}{x} : B}
      }
      {\ctxvarh{p}{P}{1} \vdash \prm{B}{p}{f}{x}{\app{f}{x}} : B}
    }
    {\vdash \lam{p}{\prm{B}{p}{f}{x}{\app{f}{x}}}
    : \fun{\underbrace{\tensor{(\fun{A}{B})}{A}}_{P}}{B}}
  \end{flalign*}
\end{frame}
\begin{frame}
  \frametitle{Vectors over semirings --- addition}
  Semiring operations (operating on annotations on individual variables) are
  lifted to vector operations (operating on contexts-worth of variables).
  \vspace{1em}
  \begin{columns}
    \begin{column}{0.5\textwidth}
      \begin{mathpar}
        \inferrule*
        {\grP\gamma \vdash M : A \\ \grQ\gamma \vdash N : B
          \\\\ \grR \subres \grP + \grQ}
        {\grR\gamma \vdash \ten{M}{N} : \tensor{A}{B}} \\
        \inferrule*
        {\grR \subres 0}
        {\grR\gamma \vdash \unit{} : \tensorOne{}}
      \end{mathpar}
    \end{column}
    \begin{column}{0.5\textwidth}
      identity, associativity, commutativity $\sim$ contexts are
      essentially multisets
    \end{column}
  \end{columns}
\end{frame}
\begin{frame}
  \frametitle{Vectors over semirings --- multiplication}
  \begin{columns}
    \begin{column}{0.5\textwidth}
      \begin{mathpar}
        \inferrule*
        {(x : A) \in \gamma \\ \grR \subres \langle x \rvert}
        {\grR\gamma \vdash x : A}
        \\
        \inferrule*
        {\grP\gamma \vdash M : A
          \\ \grR \subres \gr r \grP}
        {\grR\gamma \vdash \bang M : \excl{r}{A}}
      \end{mathpar}
    \end{column}
    \begin{column}{0.5\textwidth}
      \begin{itemize}
        \item $\langle x \rvert$ --- basis vector. The variable $x$ can be used
          once, and every other variable can be discarded.
        \item `M' for ``Multiplication'', also for ``Modality''
      \end{itemize}
    \end{column}
  \end{columns}
  \vspace{1em}
\end{frame}
\begin{frame}
  \frametitle{Vectors over semirings}
  \begin{mathpar}
    \inferrule*
    {\grR \subres 0}
    {\grR\gamma \vdash \unit{} : \tensorOne{}}
    \and
    \inferrule*
    {\grP\gamma \vdash M : A \\ \grQ\gamma \vdash N : B
      \\\\ \grR \subres \grP + \grQ}
    {\grR\gamma \vdash \ten{M}{N} : \tensor{A}{B}} \\
    \and
    \only<1>{
      \inferrule*
      {(x : A) \in \gamma \\ \grR \subres \langle x \rvert}
      {\grR\gamma \vdash x : A}
    }
    \only<2>{
      \inferrule*
      {(x : A) \sqin \grR\gamma}
      {\grR\gamma \vdash x : A}
    }
    \and
    \inferrule*
    {\grP\gamma \vdash M : A
      \\ \grR \subres \gr r \grP}
    {\grR\gamma \vdash \bang M : \excl{r}{A}}
  \end{mathpar}
  \begin{itemize}
    \item These four are the basic operations of linear algebra.
    \item $0$, $+$, and $\gr r \cdot {}$ are preserved by linear
          transformations.
    \item Notice: we can consistently add $\gr0$-use variables and maintain typing.
  \end{itemize}
\end{frame}
\begin{frame}
  \frametitle{Generic notion of linear syntax}
  Multiple premises are handled by bunched implications. \footfullcite{RPKV20}
  \begin{itemize}
    \item \(\mathfrak I~\grR\gamma := \grR \subres 0\)
    \item \((\mathcal T \sep \mathcal U)~\grR\gamma :=
      \Sigma \grP, \grQ.~(\grR \subres \grP + \grQ)
      \times \mathcal T~\grP\gamma \times \mathcal U~\grQ\gamma\)
    \item \((\gr r \cdot \mathcal T)~\grR\gamma :=
      \Sigma \grP.~(\grR \subres \gr r\grP)
      \times \mathcal T~\grP\gamma\)
    \item \((\mathcal T \wand \mathcal U)~\grP\gamma :=
      \Pi \grQ, \grR.~(\grR \subres \grP + \grQ)
      \to \mathcal T~\grQ\gamma \to \mathcal U~\grR\gamma\)
  \end{itemize}
  Example description:
  $(\excl{r}{A} \sep (\gr rA \vdash B)) \Longrightarrow B$
  \begin{itemize}
    \item \(
      \inferrule
      {\grP\gamma \vdash \excl{r}{A}
      \\ \grQ\gamma, \ctxvar{x}{A}{r} \vdash B
      \\ \grR \subres \grP + \grQ}
      {\grR\gamma \vdash B}
      \)
    \item $\sem{\mathrm{bam}} :
      \forall[~\square(\C~(\excl{r}{A})) \sep
      \square(\gr r \cdot (\V~A) \wand \C~B)
      \dotto \C~B~]$
  \end{itemize}
\end{frame}
\begin{frame}
  \frametitle{Linear Kripke}
  We're \emph{adding in} extra $\V$-values, so use $\wand$.
  \begin{flalign*}
    &\mathrm{Kripke}~\V~\C~\Gamma~A =
    \square((\Gamma \stackrel{\V}{\Rightarrow} {-}) \wand \C~A)
  \end{flalign*}

  Desiderata for environments:
  \begin{itemize}
    \item $\left({\cdot} \stackrel\V\Rightarrow {-}\right) \simeq \mathfrak I$
    \item $\left(\Gamma, \Delta \stackrel\V\Rightarrow {-}\right) \simeq
          \left(\Gamma \stackrel\V\Rightarrow {-}\right) \sep
          \left(\Delta \stackrel\V\Rightarrow {-}\right)$
    \item $\left(\gr rA \stackrel\V\Rightarrow {-}\right) \simeq
          \gr r \cdot \left(\V~A\right)$
  \end{itemize}
\end{frame}
\begin{frame}
  \frametitle{Linear environments}
  \begin{block}{Renaming}
    \[
      \underbrace{\gr1C, \gr2A, \gr4A}_{\grP\gamma}
      \only<1>{\stackrel\sqin\Rightarrow}\only<2->{\sqsubseteq}
      \underbrace{\gr6A, \gr0B, \gr1C, \gr0D}_{\grQ\delta}
    \]
    \[
      \overbrace{
        \begin{pmatrix}
          \gr6 & \gr0 & \gr1 & \gr0
        \end{pmatrix}
      }^{\grQ}
      =
      \overbrace{
        \begin{pmatrix}
          \gr1 & \gr2 & \gr4
        \end{pmatrix}
      }^{\grP}
      \begin{pmatrix}
        \gr0 & \gr0 & \gr1 & \gr0 \\
        \gr1 & \gr0 & \gr0 & \gr0 \\
        \gr1 & \gr0 & \gr0 & \gr0
      \end{pmatrix}
      \begin{matrix}
        \delta \sqni C \\
        \delta \sqni A \\
        \delta \sqni A
      \end{matrix}
  \]
  \end{block}

  \begin{columns}
    \begin{column}{0.5\textwidth}
      \begin{block}{Substitution}
        \[
          \gr1(\tensor A B) \stackrel\dashv\Rightarrow \gr1A, \gr1B
        \]
        \[
          \begin{pmatrix}
            \gr1 & \gr1
          \end{pmatrix}
          =
          \begin{pmatrix}
            \gr1
          \end{pmatrix}
          \begin{pmatrix}
            \gr1 & \gr1
          \end{pmatrix}
          \begin{matrix}
            A, B \vdash \tensor A B
          \end{matrix}
        \]
      \end{block}
    \end{column}
    \begin{column}{0.5\textwidth}
      \begin{block}{Generally}
        \begin{itemize}
          \item Pick a matrix $\gr\Psi$ such that:
          \item \( \grQ = \grP\gr\Psi \)
          \item \( \forall A, \grP\gr'.~A \sqin \grP\gr'\gamma \to
                \V~A~(\grP\gr'\gr\Psi)\delta \)
        \end{itemize}
      \end{block}
    \end{column}
  \end{columns}
\end{frame}
\begin{frame}
  \frametitle{Generic notion of linear semantics}
  \begin{itemize}
    \item Fundamental lemma of semantics:
      \[
      \inferrule*
      {\mathrm{Semantics}~\V~\C
      \\ \Gamma \stackrel\V\Rightarrow \Delta}
      {{\underbrace{A \dashv \Gamma \to \C~A~\Delta}
      _{\textrm{traversal}}}}
      \]
    \item $\mathrm{Semantics}~\V~\C$ contains:
      \begin{itemize}
        \item A proof that $\V$ is stable under structurality
        \item $\sem{\mathrm{var}} : \forall[~\V~A \dotto \C~A~]$
        \item $\sem{\mathrm{con}} :
              \forall[~\sem d\,(\mathrm{Kripke}~\V~\C)~A \dotto \C~A~]$
      \end{itemize}
    \item Traversal: similar to before, but with more algebra
  \end{itemize}
\end{frame}
\begin{frame}
  \frametitle{Showing off --- generic usage checker}
  \begin{itemize}
    \item Let $\V = {\sqin}$ and $\C~A~\gamma =
          \forall \grR.~\mathrm{List}~(\mathrm{Tm}_d~A~\grR\gamma)$.
    \item We need a way to non-deterministically invert the semiring operations
          $0$, $1$, $+$, and $\gr r \cdot {-}$.

    \item For example, $\gr3 \rightsquigarrow
          [\gr0 + \gr3, \gr1 + \gr2, \gr2 + \gr1, \gr3 + \gr0]$.
    \item Custom monadic handling of descriptions
          ($\dot1$, $\dottimes$, $\mathfrak I$, $*$, $\gr r \cdot {-}$).
    \item Traverse an unannotated term, with guess annotations for the free
          variables.
  \end{itemize}
\end{frame}
\begin{frame}
  \frametitle{Showing off --- classical linear type theory}
  \begin{itemize}
    \item Adapt Herbelin's \emph{arborescente} presentation of
          $\mu\tilde\mu$-calculus\footfullcite{herbelin-habilitation}.
    \item Where we previously had types $A$, $B$, $C$, \&c., we have
          \emph{conclusions} of the form $A~\mathrm{term}$, $A~\mathrm{coterm}$,
          or $\mathrm{command}$.
    \item (co)Variables are hypothetical (co)terms.
    \item Example rules:
      \begin{itemize}
        \item $\langle v \Vert e \rangle$:
              $A~\mathrm{term} \sep A~\mathrm{coterm}
              \Longrightarrow \mathrm{command}$
        \item $\mu\alpha.~c$:
              $(A~\mathrm{coterm} \vdash \mathrm{command})
              \Longrightarrow A~\mathrm{term}$
      \end{itemize}
  \end{itemize}
\end{frame}
\begin{frame}
  \frametitle{Conclusion}
  \begin{itemize}
    \item Adapted an intuitionistic framework to track usage information
    \item Linear metatheory in a natural deduction style
    \item First statement of linear simultaneous substitution (to my knowledge)
    \item Agda framework: \url{https://github.com/laMudri/generic-lr}
    \item Future work:
      \begin{itemize}
        \item Write the paper!
        \item Recursion/inductive types (implemented)
        \item Testing the limits of expressibility
        \item Intuitionistic requires intuitionistic, linear requires bunched ---
              what else?
      \end{itemize}
  \end{itemize}
\end{frame}

\end{document}
