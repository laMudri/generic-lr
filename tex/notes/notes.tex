\documentclass[a4paper]{article}

\usepackage{amsmath}
\usepackage{amssymb}
\usepackage{catchfilebetweentags}
\usepackage{cmll}
\usepackage[utf8]{inputenc}
\usepackage{mathpartir}
\usepackage{mathtools}
\usepackage{stmaryrd}
\usepackage{xcolor}

\def\newelims{1}
\def\multnotation{1}
\ifx\newelims\undefined
  \def\newelims{0}
\fi
\ifx\multnotation\undefined
  \def\multnotation{0}
\fi

\newcommand{\fixme}[1]{{\color{red}\underline{\em{#1}}}}
\newcommand{\fixmeBA}[1]{{\color{red}{[Bob:{\em #1}]}}}


\newcommand{\bind}[2]{{#2}\{{#1}\}}
\newcommand{\ctx}[2]{%
  \if\multnotation0%
  #1^{\resctx{#2}}%
  \else%
  \resctx{#2}#1%
  \fi
}
\newcommand{\ctxvar}[3]{%
  \if\multnotation0%
  #1 \stackrel{\rescomment{#3}}: #2%
  \else%
  \rescomment{#3}#1 : #2%
  \fi%
}
\definecolor{res}{HTML}{008000}
\definecolor{resperm}{HTML}{008000}
\newcommand{\rescomment}[1]{{\color{res}#1}}
\newcommand{\rescommentperm}[1]{{\color{resperm}#1}}
\newcommand{\resctx}[1]{\rescomment{\mathcal{#1}}}
\newcommand{\resctxperm}[1]{\rescommentperm{\mathcal{#1}}}
\newcommand{\resmat}[1]{\rescomment{#1}}


\newcommand{\ann}[2]{#1 : #2}
\newcommand{\emb}[1]{\underline{#1}}

\newcommand{\base}[0]{\iota}

\newcommand{\fun}[2]{#1 \multimap #2}
\newcommand{\lam}[2]{\lambda #1.~#2}
\newcommand{\app}[2]{#1~#2}

\newcommand{\excl}[2]{\oc_{\rescomment{#1}} #2}
\newcommand{\bang}[1]{\left[#1\right]}
\newcommand{\bm}[4]{\if\newelims0%
\operatorname{bm}_{#1}(#2, \bind{#3}{#4})%
\else%
\mathrm{let}~\bang{#3} = #2~\mathrm{in}~#4%
\fi}

\newcommand{\tensorOne}[0]{1}
\newcommand{\unit}[0]{(\mathbin{_\otimes})}
\newcommand{\del}[3]{\if\newelims0%
\operatorname{del}_{#1}(#2, #3)%
\else%
\mathrm{let}~\unit = #2~\mathrm{in}~#3%
\fi}

\newcommand{\tensor}[2]{#1 \otimes #2}
\newcommand{\ten}[2]{(#1 \mathbin{_{\otimes}} #2)}
\newcommand{\prm}[5]{\if\newelims0%
\operatorname{pm}_{#1}(#2, \bind{#3, #4}{#5})%
\else%
\mathrm{let}~\ten{#3}{#4} = #2~\mathrm{in}~#5%
\fi}

\newcommand{\withTOne}[0]{\top}
\newcommand{\eat}[0]{(\mathbin{_{\with}})}

\newcommand{\withT}[2]{#1 \with #2}
\newcommand{\wth}[2]{(#1 \mathbin{_\with} #2)}
\newcommand{\proj}[2]{\operatorname{proj}_{#1} #2}

\newcommand{\sumTZero}[0]{0}
\newcommand{\exf}[2]{\operatorname{ex-falso} #2}

\newcommand{\sumT}[2]{#1 \oplus #2}
\newcommand{\inj}[2]{\operatorname{inj}_{#1} #2}
\newcommand{\cse}[6]{\if\newelims0%
\operatorname{case}_{#1}(#2, \bind{#3}{#4}, \bind{#5}{#6})%
\else%
\mathrm{case}~#2~\mathrm{of}~\inj{L}{#3} \mapsto #4
                         ~;~ \inj{R}{#5} \mapsto #6%
\fi}

\newcommand{\listT}[1]{\operatorname{List} #1}
\newcommand{\nil}[0]{[]}
\newcommand{\cons}[2]{#1 :: #2}
\newcommand{\fold}[5]{\operatorname{fold} #1~#2~(#3,#4.~#5)}


\newcommand{\typed}[1]{\mathit{#1t}}
\newcommand{\resourced}[1]{\mathit{#1r}}


\newcommand{\lvar}{\mathrel{\mathrlap{\sqsupset}{\mathord{-}}}}
\newcommand{\sem}[1]{\left\llbracket #1 \right\rrbracket}

%\def\tobar{\mathrel{\mkern3mu  \vcenter{\hbox{$\scriptscriptstyle+$}}%
%                    \mkern-12mu{\to}}}
\newcommand\tobar{\mathrel{\ooalign{\hfil$\mapstochar\mkern5mu$\hfil\cr$\to$\cr}}}


\newcommand{\subres}{\trianglelefteq}
\newcommand{\subrctx}[2]{{#1} \subres {#2}}
\makeatletter
\newcommand{\subst}[2][]{\ext@arrow 0359\Rightarrowfill@{#1}{#2}}
\makeatother


\newenvironment{eqns}{\begin{array}{r@{\hspace{0.3em}}c@{\hspace{0.3em}}l}}{\end{array}}


\newcommand{\mat}[1]{\mathbf{#1}}
\newcommand{\vct}[1]{\mathbf{#1}}
\DeclarePairedDelimiter\basis{\langle}{\rvert}


\newcommand{\name}{\ensuremath{\lambda \resctxperm R}}


\DeclareMathOperator\kit{Kit}
\newcommand{\kitrel}{\mathbin{\blacklozenge}}


\DeclareMathOperator\id{id}
\DeclareMathOperator\inl{inl}
\DeclareMathOperator\inr{inr}


\newcommand{\zero}{\ensuremath{\rescomment 0}}
\newcommand{\linear}{\ensuremath{\rescomment 1}}
\newcommand{\unrestricted}{\ensuremath{\rescomment \omega}}

\newcommand{\unused}{\ensuremath{\rescomment{\mathit{unused}}}}
\newcommand{\true}{\ensuremath{\rescomment{\mathit{true}}}}
\newcommand{\valid}{\ensuremath{\rescomment{\mathit{valid}}}}

\newcommand\dotto{\mathrel{\dot\to}}
\newcommand\dottimes{\mathbin{\dot\times}}
\newcommand\wand{\mathrel{\mathord{-}\hspace{-0.75ex}*}}
\newcommand\sep{\mathbin{*}}


\usepackage[conor]{agda}
\CatchFileBetweenTags{\exItypes}%
{../../src/latex/Generic/Linear/Example/PaperExamples.tex}{exItypes}
\CatchFileBetweenTags{\exIlabels}%
{../../src/latex/Generic/Linear/Example/PaperExamples.tex}{exIlabels}
\CatchFileBetweenTags{\exIfunrules}%
{../../src/latex/Generic/Linear/Example/PaperExamples.tex}{exIfunrules}
\CatchFileBetweenTags{\exIsumrules}%
{../../src/latex/Generic/Linear/Example/PaperExamples.tex}{exIsumrules}

\CatchFileBetweenTags{\Premises}%
{../../src/latex/Generic/Linear/Syntax.tex}{Premises}
\CatchFileBetweenTags{\Rule}%
{../../src/latex/Generic/Linear/Syntax.tex}{Rule}
\CatchFileBetweenTags{\System}%
{../../src/latex/Generic/Linear/Syntax.tex}{System}

\CatchFileBetweenTags{\semp}%
{../../src/latex/Generic/Linear/Syntax/Interpretation.tex}{semp}
\CatchFileBetweenTags{\semr}%
{../../src/latex/Generic/Linear/Syntax/Interpretation.tex}{semr}
\CatchFileBetweenTags{\sems}%
{../../src/latex/Generic/Linear/Syntax/Interpretation.tex}{sems}

\usepackage{newunicodechar}
\newunicodechar{λ}{\ensuremath{\mathnormal\lambda}}
\newunicodechar{→}{\ensuremath{\mathnormal\to}}
\newunicodechar{∀}{\ensuremath{\mathnormal\forall}}
\newunicodechar{ι}{\ensuremath{\mathnormal\iota}}
\newunicodechar{·}{\ensuremath{\mathnormal\cdot}}
\newunicodechar{⊸}{\ensuremath{\mathnormal\multimap}}
\newunicodechar{⊕}{\ensuremath{\mathnormal\oplus}}
\newunicodechar{─}{\textrm{---}}
\newunicodechar{ᶜ}{\ensuremath{\mathnormal{^c}}}
\newunicodechar{⊢}{\ensuremath{\mathnormal\vdash}}
\newunicodechar{∧}{\ensuremath{\mathnormal\wedge}}
\newunicodechar{⟦}{\ensuremath{\mathnormal\llbracket}}
\newunicodechar{⟧}{\ensuremath{\mathnormal\rrbracket}}
\newunicodechar{∩}{\ensuremath{\mathnormal\cap}}
\newunicodechar{✴}{\ensuremath{\mathnormal*}}
\newunicodechar{ℓ}{\ensuremath{\mathnormal\ell}}
\newunicodechar{Γ}{\ensuremath{\mathnormal\Gamma}}
\newunicodechar{Σ}{\ensuremath{\mathnormal\Sigma}}
\newunicodechar{∈}{\ensuremath{\mathnormal\in}}
\newunicodechar{′}{\ensuremath{\mathnormal'}}
\newunicodechar{≡}{\ensuremath{\mathnormal\equiv}}
\newunicodechar{⊤}{\ensuremath{\mathnormal\top}}
\newunicodechar{⊥}{\ensuremath{\mathnormal\bot}}
\newunicodechar{⇒}{\ensuremath{\mathnormal\Rightarrow}}
\newunicodechar{▹}{\ensuremath{\mathnormal\triangleright}}
\newunicodechar{₁}{\ensuremath{\mathnormal{_1}}}
\newunicodechar{ℑ}{\ensuremath{\mathfrak I}}
\newunicodechar{□}{\ensuremath{\mathnormal\Box}}

\begin{document}
\section{Building up}

We assume that a type system comprises a set of (unparametrised) rules, each
of which has a conclusion and several premises containing subterms.
The primary investigation of this work is into what form the premises can take
while maintaining useful features of syntax.
We shall start with simple forms, allowing just for multiple subterms, and
then build on resource-sensitive bunches, variable binding, and modalities.

\System{}
\Rule{}

Given some way \AgdaFunction{⟦\AgdaUnderscore{}⟧p} of interpreting
\AgdaDatatype{Premises} in a \AgdaRecord{Ctx}, we can interpret a
\AgdaDatatype{Rule} against a conclusion and a context by checking that its
stated conclusion matches and then interpreting its premises.
Then, the entire \AgdaDatatype{System} can be interpreted by picking a rule
label (including parameters) \AgdaBound{l} and interpreting the selected rule
\AgdaBound{rs}\AgdaSpace{}\AgdaBound{l}.

\semr{}
\sems{}

\subsection{The language of Cartesian products}

\ldots

\section{Bunched functions}

Suppose we have a left $R$-semimodule $M$.
Then we can define the following connectives on indexed type families.

\begin{align*}
  \top~z &:= \top \\
  (P \wedge Q)~z &:= P~z \wedge Q~z \\
  (P \to Q)~z &:= P~z \to Q~z \\
  I~z &:= z \subres 0 \\
  (P * Q)~z &:= \exists x,y.~z \subres x+y \wedge P~x \wedge Q~y \\
  (P \wand Q)~y &:= \forall x,z.~z \subres x+y \to P~x \to Q~z \\
  (r \cdot P)~z &:= \exists x.~z \subres rx \wedge P~x
\end{align*}

\paragraph{Agda definitions}
To follow.

The first six of these connectives are somewhat standard, the first three
corresponding to the usual sharing connectives and the second three
corresponding to the usual separating connectives.
The final connective, $r \cdot P$, is new.
We choose $\exists$ rather than $\forall$ because scaling will usually occur in
premises, i.e., to the left of arrows, and we want to avoid anything
higher-order.

\section{Generic syntax}

Rules $R$, premises $P$ and $Q$.

\begin{align*}
  R &::= P/A \\
  P,Q &::= \ctx \Theta R \vdash A \mid \top \mid P \wedge Q \mid I \mid P * Q
        \mid \rescomment r \cdot P
\end{align*}

Example rules:

\begin{itemize}
  \item With-introduction: ${\vdash A} \wedge {\vdash B} / \withT{A}{B}$
  \item Annotated arrow introduction:
    ${\rescomment rA \vdash B} / \fun{\rescomment rA}{B}$
  \item Annotated arrow elimination:
    ${\vdash \fun{\rescomment rA}{B}} * \rescomment r \cdot {\vdash A} / B$
  \item Cases of annotated sum:
    ${\vdash \sumT{\rescomment rA}{\rescomment sB}}
    * ({\rescomment rA \vdash C} \wedge {\rescomment sB \vdash C}) / C$
\end{itemize}

\paragraph{Agda example}

We start by defining our types \AgdaDatatype{Ty}, which in this case will be the
conclusions of our judgements.
\AgdaFunction{Ann} is the type of usage annotations, which for now is an
arbitrary skew semiring.
We also have a type of rule labels \AgdaDatatype{`Sys}, which specifies all of
the parameters of a typing rule except for its conclusion and premises.
For example, in the \AgdaInductiveConstructor{`case} label, we need to fill in
the two summands (types \AgdaBound{A} and \AgdaBound{B}), as well as the return
type \AgdaBound{C}.
The actual form of the rule will be specified later.

\exItypes{} \exIlabels{}

We are then ready to build a system \AgdaFunction{Sys} with conclusions
\AgdaDatatype{Ty}, usage annotations \AgdaFunction{Ann}, and labels
\AgdaDatatype{`Sys}.
The rule corresponding to each label is given in the following.

The \AgdaInductiveConstructor{`lam} rule has one premise, which binds one
variable with usage annotation \AgdaBound{r} and type \AgdaBound{A}.
The \AgdaInductiveConstructor{`app} rule has two premises combined with
separating conjunction.
Furthermore, the second premise (corresponding to the argument of the function)
is subject to scaling by \AgdaBound{r}.
Intuitively, this description means that we must have enough of each variable to
use some in building a function, and then use some more \AgdaBound{r} times to
build enough arguments.

\exIfunrules{}

In the rules for sums, we use an alternative notation for rule descriptions,
with \AgdaFunction{{---}{---}} being an infix version of
\AgdaInductiveConstructor{rule}.
The introduction rules \AgdaInductiveConstructor{`inl} and
\AgdaInductiveConstructor{`inr} are straightforward --- each has one
premise which binds no new variables.
The \AgdaInductiveConstructor{`case} rule is more complicated.
The premises are first split into the eliminand and the continuations, using the
separating conjunction we saw for \AgdaInductiveConstructor{`app}.
However, the continuations are connected via the sharing conjunction, reflecting
the fact that in any given run of the program, only one of the branches will be
taken, so usages from each individually should be added to usages of the
eliminand.

\exIsumrules{}

\paragraph{Interpretation as syntax}
We can say something like ``premise connectives are interpreted as the
corresponding bunched connectives, where appropriate''.

\begin{align*}
  \sem{P/A} &:= \sem P \to {- \vdash A} \\
  \sem{\ctx \Theta R \vdash A} &:= {-, \ctx \Theta R \vdash A} \\
  \sem{\top} &:= \top \\
  \sem{P \wedge Q} &:= \sem P \wedge \sem Q \\
  \sem{I} &:= I \\
  \sem{P * Q} &:= \sem P * \sem Q \\
  &\ldots
\end{align*}

\paragraph{Agda version}
The interpretation of a system is the selection of a rule, together with the
interpretation of that rule.

\sems{}

The interpretation of a rule is a check that the rule targets the desired
conclusion, together with the interpretation of the rules premises.

\semr{}

For the premises, connectives are interpreted in the obvious way.
Premises can ask for subterms via the $\AgdaInductiveConstructor{\_`⊢\_}$
constructor, which supplies the new variables (in particular, each variable's
type and usage annotation) \AgdaBound{Γ} and the desired conclusion
\AgdaBound{A} of the subterm.

\semp{}

\section{Environments}

An environment is a linear map from plain variables in one context to semantic
variables in another.

\section{Thinnings}

A thinning is an environment of syntactic variables.
In effect, thinning allows for permutation and slackening of existing
variables, and introduction of new discardable variables at any position.

Given this definition, $\square$ and \texttt{Thinnable} are as in AACMM.
\texttt{Kripke} is modified to use separating implication.

\end{document}
